\section{Introdução} \label{section: introduction}

Neste tutorial vamos construir uma aplicação simples para gerir os produtos da ementa de um restaurante. O foco estará no desenvolvimento de uma solução backend em Node.js \cite{nodejs_nodejs_nodate} utilizando a \textit{framework} \textbf{fastify} \cite{noauthor_fastify_nodate} para a criação das rotas responsáveis por gerir todas as operações CRUD \cite{noauthor_what_nodate} dos nossos produtos. A nível de persistência será utilizada a biblioteca \textbf{mongoose} \cite{noauthor_mongoose_nodate} para simplificar a modelação de objetos de uma base de dados MongoDB \cite{noauthor_mongodb_nodate}.

\subsection{Pré-requisitos}

Para tirar o máximo partido deste tutorial, deverá compreender o fundamental dos princípios REST \cite{noauthor_what_nodate-1}, bem como conhecer o básico de MongoDB e de uma \textit{framework} web de Node.js, como o Express.js \cite{noauthor_express_nodate}.

Será utilizada a sintaxe do ES6 \cite{noauthor_ecmascript_nodate}, pelo qual se espera o conhecimento de conceitos como desestruturação, funções \textit{arrow} e outros. De qualquer modo, todo o código o apresentado será explicado em detalhe.

De modo geral, os princípios demonstrados neste projeto não são particularmente complexos, pelo que este tutorial pode ser considerado como uma introdução às tecnologias listadas acima.
